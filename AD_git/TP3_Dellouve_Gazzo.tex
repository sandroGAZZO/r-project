% Options for packages loaded elsewhere
\PassOptionsToPackage{unicode}{hyperref}
\PassOptionsToPackage{hyphens}{url}
%
\documentclass[
]{article}
\usepackage{lmodern}
\usepackage{amssymb,amsmath}
\usepackage{ifxetex,ifluatex}
\ifnum 0\ifxetex 1\fi\ifluatex 1\fi=0 % if pdftex
  \usepackage[T1]{fontenc}
  \usepackage[utf8]{inputenc}
  \usepackage{textcomp} % provide euro and other symbols
\else % if luatex or xetex
  \usepackage{unicode-math}
  \defaultfontfeatures{Scale=MatchLowercase}
  \defaultfontfeatures[\rmfamily]{Ligatures=TeX,Scale=1}
\fi
% Use upquote if available, for straight quotes in verbatim environments
\IfFileExists{upquote.sty}{\usepackage{upquote}}{}
\IfFileExists{microtype.sty}{% use microtype if available
  \usepackage[]{microtype}
  \UseMicrotypeSet[protrusion]{basicmath} % disable protrusion for tt fonts
}{}
\makeatletter
\@ifundefined{KOMAClassName}{% if non-KOMA class
  \IfFileExists{parskip.sty}{%
    \usepackage{parskip}
  }{% else
    \setlength{\parindent}{0pt}
    \setlength{\parskip}{6pt plus 2pt minus 1pt}}
}{% if KOMA class
  \KOMAoptions{parskip=half}}
\makeatother
\usepackage{xcolor}
\IfFileExists{xurl.sty}{\usepackage{xurl}}{} % add URL line breaks if available
\IfFileExists{bookmark.sty}{\usepackage{bookmark}}{\usepackage{hyperref}}
\hypersetup{
  pdftitle={TP3 AD: Dellouve \& Gazzo},
  hidelinks,
  pdfcreator={LaTeX via pandoc}}
\urlstyle{same} % disable monospaced font for URLs
\usepackage[margin=1in]{geometry}
\usepackage{color}
\usepackage{fancyvrb}
\newcommand{\VerbBar}{|}
\newcommand{\VERB}{\Verb[commandchars=\\\{\}]}
\DefineVerbatimEnvironment{Highlighting}{Verbatim}{commandchars=\\\{\}}
% Add ',fontsize=\small' for more characters per line
\usepackage{framed}
\definecolor{shadecolor}{RGB}{248,248,248}
\newenvironment{Shaded}{\begin{snugshade}}{\end{snugshade}}
\newcommand{\AlertTok}[1]{\textcolor[rgb]{0.94,0.16,0.16}{#1}}
\newcommand{\AnnotationTok}[1]{\textcolor[rgb]{0.56,0.35,0.01}{\textbf{\textit{#1}}}}
\newcommand{\AttributeTok}[1]{\textcolor[rgb]{0.77,0.63,0.00}{#1}}
\newcommand{\BaseNTok}[1]{\textcolor[rgb]{0.00,0.00,0.81}{#1}}
\newcommand{\BuiltInTok}[1]{#1}
\newcommand{\CharTok}[1]{\textcolor[rgb]{0.31,0.60,0.02}{#1}}
\newcommand{\CommentTok}[1]{\textcolor[rgb]{0.56,0.35,0.01}{\textit{#1}}}
\newcommand{\CommentVarTok}[1]{\textcolor[rgb]{0.56,0.35,0.01}{\textbf{\textit{#1}}}}
\newcommand{\ConstantTok}[1]{\textcolor[rgb]{0.00,0.00,0.00}{#1}}
\newcommand{\ControlFlowTok}[1]{\textcolor[rgb]{0.13,0.29,0.53}{\textbf{#1}}}
\newcommand{\DataTypeTok}[1]{\textcolor[rgb]{0.13,0.29,0.53}{#1}}
\newcommand{\DecValTok}[1]{\textcolor[rgb]{0.00,0.00,0.81}{#1}}
\newcommand{\DocumentationTok}[1]{\textcolor[rgb]{0.56,0.35,0.01}{\textbf{\textit{#1}}}}
\newcommand{\ErrorTok}[1]{\textcolor[rgb]{0.64,0.00,0.00}{\textbf{#1}}}
\newcommand{\ExtensionTok}[1]{#1}
\newcommand{\FloatTok}[1]{\textcolor[rgb]{0.00,0.00,0.81}{#1}}
\newcommand{\FunctionTok}[1]{\textcolor[rgb]{0.00,0.00,0.00}{#1}}
\newcommand{\ImportTok}[1]{#1}
\newcommand{\InformationTok}[1]{\textcolor[rgb]{0.56,0.35,0.01}{\textbf{\textit{#1}}}}
\newcommand{\KeywordTok}[1]{\textcolor[rgb]{0.13,0.29,0.53}{\textbf{#1}}}
\newcommand{\NormalTok}[1]{#1}
\newcommand{\OperatorTok}[1]{\textcolor[rgb]{0.81,0.36,0.00}{\textbf{#1}}}
\newcommand{\OtherTok}[1]{\textcolor[rgb]{0.56,0.35,0.01}{#1}}
\newcommand{\PreprocessorTok}[1]{\textcolor[rgb]{0.56,0.35,0.01}{\textit{#1}}}
\newcommand{\RegionMarkerTok}[1]{#1}
\newcommand{\SpecialCharTok}[1]{\textcolor[rgb]{0.00,0.00,0.00}{#1}}
\newcommand{\SpecialStringTok}[1]{\textcolor[rgb]{0.31,0.60,0.02}{#1}}
\newcommand{\StringTok}[1]{\textcolor[rgb]{0.31,0.60,0.02}{#1}}
\newcommand{\VariableTok}[1]{\textcolor[rgb]{0.00,0.00,0.00}{#1}}
\newcommand{\VerbatimStringTok}[1]{\textcolor[rgb]{0.31,0.60,0.02}{#1}}
\newcommand{\WarningTok}[1]{\textcolor[rgb]{0.56,0.35,0.01}{\textbf{\textit{#1}}}}
\usepackage{graphicx,grffile}
\makeatletter
\def\maxwidth{\ifdim\Gin@nat@width>\linewidth\linewidth\else\Gin@nat@width\fi}
\def\maxheight{\ifdim\Gin@nat@height>\textheight\textheight\else\Gin@nat@height\fi}
\makeatother
% Scale images if necessary, so that they will not overflow the page
% margins by default, and it is still possible to overwrite the defaults
% using explicit options in \includegraphics[width, height, ...]{}
\setkeys{Gin}{width=\maxwidth,height=\maxheight,keepaspectratio}
% Set default figure placement to htbp
\makeatletter
\def\fps@figure{htbp}
\makeatother
\setlength{\emergencystretch}{3em} % prevent overfull lines
\providecommand{\tightlist}{%
  \setlength{\itemsep}{0pt}\setlength{\parskip}{0pt}}
\setcounter{secnumdepth}{-\maxdimen} % remove section numbering

\title{TP3 AD: Dellouve \& Gazzo}
\author{}
\date{\vspace{-2.5em}}

\begin{document}
\maketitle

\#Exercice 3

\#Question 1

On commence par définir nos données.

\begin{Shaded}
\begin{Highlighting}[]
\KeywordTok{options}\NormalTok{(}\DataTypeTok{tinytex.verbose =} \OtherTok{TRUE}\NormalTok{)}
\KeywordTok{library}\NormalTok{(}\StringTok{"FactoMineR"}\NormalTok{)}
\KeywordTok{library}\NormalTok{(}\StringTok{"factoextra"}\NormalTok{)}
\end{Highlighting}
\end{Shaded}

\begin{verbatim}
## Loading required package: ggplot2
\end{verbatim}

\begin{verbatim}
## Welcome! Want to learn more? See two factoextra-related books at https://goo.gl/ve3WBa
\end{verbatim}

\begin{Shaded}
\begin{Highlighting}[]
\NormalTok{A<-}\KeywordTok{matrix}\NormalTok{(}\KeywordTok{c}\NormalTok{(}\DecValTok{1}\NormalTok{,}\DecValTok{0}\NormalTok{,}\DecValTok{2}\NormalTok{,}\DecValTok{1}\NormalTok{,}\DecValTok{0}\NormalTok{,}\DecValTok{1}\NormalTok{,}\DecValTok{3}\NormalTok{,}\DecValTok{2}\NormalTok{,}\DecValTok{4}\NormalTok{,}\DecValTok{3}\NormalTok{,}\DecValTok{3}\NormalTok{,}\DecValTok{4}\NormalTok{,}\DecValTok{4}\NormalTok{,}\DecValTok{4}\NormalTok{),}\DataTypeTok{ncol=}\DecValTok{2}\NormalTok{, }\DataTypeTok{byrow=}\OtherTok{TRUE}\NormalTok{) }\CommentTok{# on défini X}
\NormalTok{X<-}\KeywordTok{as.data.frame}\NormalTok{(A) }\CommentTok{# changement de format}
\KeywordTok{colnames}\NormalTok{(X)<-}\KeywordTok{c}\NormalTok{(}\StringTok{"var1"}\NormalTok{, }\StringTok{"var2"}\NormalTok{) }\CommentTok{# on nomme les colonnes}
\KeywordTok{rownames}\NormalTok{(X)<-}\KeywordTok{c}\NormalTok{(}\StringTok{"w1"}\NormalTok{, }\StringTok{"w2"}\NormalTok{, }\StringTok{"w3"}\NormalTok{,}\StringTok{"w4"}\NormalTok{,}\StringTok{"w5"}\NormalTok{,}\StringTok{"w6"}\NormalTok{,}\StringTok{"w7"}\NormalTok{) }\CommentTok{# on nomme les lignes}
\NormalTok{X }\CommentTok{# on affiche X}
\end{Highlighting}
\end{Shaded}

\begin{verbatim}
##    var1 var2
## w1    1    0
## w2    2    1
## w3    0    1
## w4    3    2
## w5    4    3
## w6    3    4
## w7    4    4
\end{verbatim}

On cherche la partition en deux classes donnée par la classification
hiérarchique (CAH) ayant comme critère d'agrégation la distance du saut
de Ward.

\begin{Shaded}
\begin{Highlighting}[]
\CommentTok{# ETAPE PAR ETAPE, ETAPE 1}


\CommentTok{# Distance euclidienne}
\NormalTok{dist <-}\StringTok{ }\KeywordTok{dist}\NormalTok{(X, }\DataTypeTok{method =} \StringTok{"euclidean"}\NormalTok{)}
\NormalTok{dist}
\end{Highlighting}
\end{Shaded}

\begin{verbatim}
##          w1       w2       w3       w4       w5       w6
## w2 1.414214                                             
## w3 1.414214 2.000000                                    
## w4 2.828427 1.414214 3.162278                           
## w5 4.242641 2.828427 4.472136 1.414214                  
## w6 4.472136 3.162278 4.242641 2.000000 1.414214         
## w7 5.000000 3.605551 5.000000 2.236068 1.000000 1.000000
\end{verbatim}

\begin{Shaded}
\begin{Highlighting}[]
\CommentTok{##La matrice des distances au carré}
\NormalTok{dist}\OperatorTok{^}\DecValTok{2}
\end{Highlighting}
\end{Shaded}

\begin{verbatim}
##    w1 w2 w3 w4 w5 w6
## w2  2               
## w3  2  4            
## w4  8  2 10         
## w5 18  8 20  2      
## w6 20 10 18  4  2   
## w7 25 13 25  5  1  1
\end{verbatim}

\begin{Shaded}
\begin{Highlighting}[]
\CommentTok{#La matrice des sauts de ward avec la distance au carré est:}
\NormalTok{ward1<-(dist}\OperatorTok{^}\DecValTok{2}\NormalTok{)}\OperatorTok{/}\DecValTok{8}
\NormalTok{ward1}
\end{Highlighting}
\end{Shaded}

\begin{verbatim}
##       w1    w2    w3    w4    w5    w6
## w2 0.250                              
## w3 0.250 0.500                        
## w4 1.000 0.250 1.250                  
## w5 2.250 1.000 2.500 0.250            
## w6 2.500 1.250 2.250 0.500 0.250      
## w7 3.125 1.625 3.125 0.625 0.125 0.125
\end{verbatim}

\begin{Shaded}
\begin{Highlighting}[]
\CommentTok{##La plus petite perte d'inertie inter est entre w5 et W7}
\NormalTok{pertemin1=}\KeywordTok{min}\NormalTok{(ward1)}
\KeywordTok{fviz_dist}\NormalTok{(dist, }\DataTypeTok{lab_size =} \DecValTok{8}\NormalTok{)}
\end{Highlighting}
\end{Shaded}

\includegraphics{TP3_Dellouve_Gazzo_files/figure-latex/unnamed-chunk-2-1.pdf}

\begin{Shaded}
\begin{Highlighting}[]
\CommentTok{#ETAPE 2}

\NormalTok{A<-}\KeywordTok{matrix}\NormalTok{(}\KeywordTok{c}\NormalTok{(}\DecValTok{1}\NormalTok{,}\DecValTok{0}\NormalTok{,}\DecValTok{2}\NormalTok{,}\DecValTok{1}\NormalTok{,}\DecValTok{0}\NormalTok{,}\DecValTok{1}\NormalTok{,}\DecValTok{3}\NormalTok{,}\DecValTok{2}\NormalTok{,}\DecValTok{3}\NormalTok{,}\DecValTok{4}\NormalTok{,}\FloatTok{3.5}\NormalTok{,}\DecValTok{4}\NormalTok{),}\DataTypeTok{ncol=}\DecValTok{2}\NormalTok{, }\DataTypeTok{byrow=}\OtherTok{TRUE}\NormalTok{) }\CommentTok{# on défini X}
\NormalTok{X<-}\KeywordTok{as.data.frame}\NormalTok{(A) }\CommentTok{# changement de format}
\KeywordTok{colnames}\NormalTok{(X)<-}\KeywordTok{c}\NormalTok{(}\StringTok{"var1"}\NormalTok{, }\StringTok{"var2"}\NormalTok{) }\CommentTok{# on nomme les colonnes}
\KeywordTok{rownames}\NormalTok{(X)<-}\KeywordTok{c}\NormalTok{(}\StringTok{"w1"}\NormalTok{, }\StringTok{"w2"}\NormalTok{, }\StringTok{"w3"}\NormalTok{,}\StringTok{"w4"}\NormalTok{,}\StringTok{"w6"}\NormalTok{,}\StringTok{"g57"}\NormalTok{) }\CommentTok{# on nomme les lignes}
\NormalTok{X }\CommentTok{# on affiche X}
\end{Highlighting}
\end{Shaded}

\begin{verbatim}
##     var1 var2
## w1   1.0    0
## w2   2.0    1
## w3   0.0    1
## w4   3.0    2
## w6   3.0    4
## g57  3.5    4
\end{verbatim}

\begin{Shaded}
\begin{Highlighting}[]
\CommentTok{# Distance euclidienne}
\NormalTok{dist <-}\StringTok{ }\KeywordTok{dist}\NormalTok{(X, }\DataTypeTok{method =} \StringTok{"euclidean"}\NormalTok{)}
\NormalTok{dist}
\end{Highlighting}
\end{Shaded}

\begin{verbatim}
##           w1       w2       w3       w4       w6
## w2  1.414214                                    
## w3  1.414214 2.000000                           
## w4  2.828427 1.414214 3.162278                  
## w6  4.472136 3.162278 4.242641 2.000000         
## g57 4.716991 3.354102 4.609772 2.061553 0.500000
\end{verbatim}

\begin{Shaded}
\begin{Highlighting}[]
\CommentTok{##La matrice des distances au carré}
\NormalTok{dist}\OperatorTok{^}\DecValTok{2}
\end{Highlighting}
\end{Shaded}

\begin{verbatim}
##        w1    w2    w3    w4    w6
## w2   2.00                        
## w3   2.00  4.00                  
## w4   8.00  2.00 10.00            
## w6  20.00 10.00 18.00  4.00      
## g57 22.25 11.25 21.25  4.25  0.25
\end{verbatim}

\begin{Shaded}
\begin{Highlighting}[]
\CommentTok{#La matrice des sauts de ward avec la distance au carré est:}
\NormalTok{ward1<-(dist}\OperatorTok{^}\DecValTok{2}\NormalTok{)}\OperatorTok{/}\DecValTok{8}
\NormalTok{ward1}
\end{Highlighting}
\end{Shaded}

\begin{verbatim}
##          w1      w2      w3      w4      w6
## w2  0.25000                                
## w3  0.25000 0.50000                        
## w4  1.00000 0.25000 1.25000                
## w6  2.50000 1.25000 2.25000 0.50000        
## g57 2.78125 1.40625 2.65625 0.53125 0.03125
\end{verbatim}

\begin{Shaded}
\begin{Highlighting}[]
\CommentTok{##La plus petite perte d'inertie inter est entre w6 et g57}
\NormalTok{pertemin1=}\KeywordTok{min}\NormalTok{(ward1)}
\KeywordTok{fviz_dist}\NormalTok{(dist, }\DataTypeTok{lab_size =} \DecValTok{8}\NormalTok{)}
\end{Highlighting}
\end{Shaded}

\includegraphics{TP3_Dellouve_Gazzo_files/figure-latex/unnamed-chunk-2-2.pdf}

\begin{Shaded}
\begin{Highlighting}[]
\CommentTok{#ETAPE 2}

\NormalTok{A<-}\KeywordTok{matrix}\NormalTok{(}\KeywordTok{c}\NormalTok{(}\DecValTok{1}\NormalTok{,}\DecValTok{0}\NormalTok{,}\DecValTok{2}\NormalTok{,}\DecValTok{1}\NormalTok{,}\DecValTok{0}\NormalTok{,}\DecValTok{1}\NormalTok{,}\DecValTok{3}\NormalTok{,}\DecValTok{2}\NormalTok{,}\FloatTok{11.}\OperatorTok{/}\DecValTok{3}\NormalTok{,}\FloatTok{11.}\OperatorTok{/}\DecValTok{3}\NormalTok{),}\DataTypeTok{ncol=}\DecValTok{2}\NormalTok{, }\DataTypeTok{byrow=}\OtherTok{TRUE}\NormalTok{) }\CommentTok{# on défini X}
\NormalTok{X<-}\KeywordTok{as.data.frame}\NormalTok{(A) }\CommentTok{# changement de format}
\KeywordTok{colnames}\NormalTok{(X)<-}\KeywordTok{c}\NormalTok{(}\StringTok{"var1"}\NormalTok{, }\StringTok{"var2"}\NormalTok{) }\CommentTok{# on nomme les colonnes}
\KeywordTok{rownames}\NormalTok{(X)<-}\KeywordTok{c}\NormalTok{(}\StringTok{"w1"}\NormalTok{, }\StringTok{"w2"}\NormalTok{, }\StringTok{"w3"}\NormalTok{,}\StringTok{"w4"}\NormalTok{,}\StringTok{"g567"}\NormalTok{) }\CommentTok{# on nomme les lignes}
\NormalTok{X }\CommentTok{# on affiche X}
\end{Highlighting}
\end{Shaded}

\begin{verbatim}
##          var1     var2
## w1   1.000000 0.000000
## w2   2.000000 1.000000
## w3   0.000000 1.000000
## w4   3.000000 2.000000
## g567 3.666667 3.666667
\end{verbatim}

\begin{Shaded}
\begin{Highlighting}[]
\CommentTok{# Distance euclidienne}
\NormalTok{dist <-}\StringTok{ }\KeywordTok{dist}\NormalTok{(X, }\DataTypeTok{method =} \StringTok{"euclidean"}\NormalTok{)}
\NormalTok{dist}
\end{Highlighting}
\end{Shaded}

\begin{verbatim}
##            w1       w2       w3       w4
## w2   1.414214                           
## w3   1.414214 2.000000                  
## w4   2.828427 1.414214 3.162278         
## g567 4.533824 3.144660 4.533824 1.795055
\end{verbatim}

\begin{Shaded}
\begin{Highlighting}[]
\CommentTok{##La matrice des distances au carré}
\NormalTok{dist}\OperatorTok{^}\DecValTok{2}
\end{Highlighting}
\end{Shaded}

\begin{verbatim}
##             w1        w2        w3        w4
## w2    2.000000                              
## w3    2.000000  4.000000                    
## w4    8.000000  2.000000 10.000000          
## g567 20.555556  9.888889 20.555556  3.222222
\end{verbatim}

\begin{Shaded}
\begin{Highlighting}[]
\CommentTok{#La matrice des sauts de ward avec la distance au carré est:}
\NormalTok{ward1<-(dist}\OperatorTok{^}\DecValTok{2}\NormalTok{)}\OperatorTok{/}\DecValTok{8}
\NormalTok{ward1}
\end{Highlighting}
\end{Shaded}

\begin{verbatim}
##             w1        w2        w3        w4
## w2   0.2500000                              
## w3   0.2500000 0.5000000                    
## w4   1.0000000 0.2500000 1.2500000          
## g567 2.5694444 1.2361111 2.5694444 0.4027778
\end{verbatim}

\begin{Shaded}
\begin{Highlighting}[]
\CommentTok{##La plus petite perte d'inertie inter est entre w1 et w2}
\CommentTok{# on avait aussi w1 w3 et w4 w2}
\NormalTok{pertemin1=}\KeywordTok{min}\NormalTok{(ward1)}
\KeywordTok{fviz_dist}\NormalTok{(dist, }\DataTypeTok{lab_size =} \DecValTok{8}\NormalTok{)}
\end{Highlighting}
\end{Shaded}

\includegraphics{TP3_Dellouve_Gazzo_files/figure-latex/unnamed-chunk-2-3.pdf}

\begin{Shaded}
\begin{Highlighting}[]
\CommentTok{#ETAPE 3}

\NormalTok{A<-}\KeywordTok{matrix}\NormalTok{(}\KeywordTok{c}\NormalTok{(}\FloatTok{1.5}\NormalTok{,}\FloatTok{0.5}\NormalTok{,}\DecValTok{0}\NormalTok{,}\DecValTok{1}\NormalTok{,}\DecValTok{3}\NormalTok{,}\DecValTok{2}\NormalTok{,}\FloatTok{11.}\OperatorTok{/}\DecValTok{3}\NormalTok{,}\FloatTok{11.}\OperatorTok{/}\DecValTok{3}\NormalTok{),}\DataTypeTok{ncol=}\DecValTok{2}\NormalTok{, }\DataTypeTok{byrow=}\OtherTok{TRUE}\NormalTok{) }\CommentTok{# on défini X}
\NormalTok{X<-}\KeywordTok{as.data.frame}\NormalTok{(A) }\CommentTok{# changement de format}
\KeywordTok{colnames}\NormalTok{(X)<-}\KeywordTok{c}\NormalTok{(}\StringTok{"var1"}\NormalTok{, }\StringTok{"var2"}\NormalTok{) }\CommentTok{# on nomme les colonnes}
\KeywordTok{rownames}\NormalTok{(X)<-}\KeywordTok{c}\NormalTok{(}\StringTok{"g12"}\NormalTok{, }\StringTok{"w3"}\NormalTok{,}\StringTok{"w4"}\NormalTok{,}\StringTok{"g567"}\NormalTok{) }\CommentTok{# on nomme les lignes}
\NormalTok{X }\CommentTok{# on affiche X}
\end{Highlighting}
\end{Shaded}

\begin{verbatim}
##          var1     var2
## g12  1.500000 0.500000
## w3   0.000000 1.000000
## w4   3.000000 2.000000
## g567 3.666667 3.666667
\end{verbatim}

\begin{Shaded}
\begin{Highlighting}[]
\CommentTok{# Distance euclidienne}
\NormalTok{dist <-}\StringTok{ }\KeywordTok{dist}\NormalTok{(X, }\DataTypeTok{method =} \StringTok{"euclidean"}\NormalTok{)}
\NormalTok{dist}
\end{Highlighting}
\end{Shaded}

\begin{verbatim}
##           g12       w3       w4
## w3   1.581139                  
## w4   2.121320 3.162278         
## g567 3.836955 4.533824 1.795055
\end{verbatim}

\begin{Shaded}
\begin{Highlighting}[]
\CommentTok{##La matrice des distances au carré}
\NormalTok{dist}\OperatorTok{^}\DecValTok{2}
\end{Highlighting}
\end{Shaded}

\begin{verbatim}
##            g12        w3        w4
## w3    2.500000                    
## w4    4.500000 10.000000          
## g567 14.722222 20.555556  3.222222
\end{verbatim}

\begin{Shaded}
\begin{Highlighting}[]
\CommentTok{#La matrice des sauts de ward avec la distance au carré est:}
\NormalTok{ward1<-(dist}\OperatorTok{^}\DecValTok{2}\NormalTok{)}\OperatorTok{/}\DecValTok{8}
\NormalTok{ward1}
\end{Highlighting}
\end{Shaded}

\begin{verbatim}
##            g12        w3        w4
## w3   0.3125000                    
## w4   0.5625000 1.2500000          
## g567 1.8402778 2.5694444 0.4027778
\end{verbatim}

\begin{Shaded}
\begin{Highlighting}[]
\CommentTok{##La plus petite perte d'inertie inter est entre g12 et w3}
\NormalTok{pertemin1=}\KeywordTok{min}\NormalTok{(ward1)}
\KeywordTok{fviz_dist}\NormalTok{(dist, }\DataTypeTok{lab_size =} \DecValTok{8}\NormalTok{)}
\end{Highlighting}
\end{Shaded}

\includegraphics{TP3_Dellouve_Gazzo_files/figure-latex/unnamed-chunk-2-4.pdf}

\begin{Shaded}
\begin{Highlighting}[]
\CommentTok{#ETAPE 4}

\NormalTok{A<-}\KeywordTok{matrix}\NormalTok{(}\KeywordTok{c}\NormalTok{(}\DecValTok{1}\NormalTok{,}\FloatTok{2.}\OperatorTok{/}\DecValTok{3}\NormalTok{,}\DecValTok{3}\NormalTok{,}\DecValTok{2}\NormalTok{,}\FloatTok{11.}\OperatorTok{/}\DecValTok{3}\NormalTok{,}\FloatTok{11.}\OperatorTok{/}\DecValTok{3}\NormalTok{),}\DataTypeTok{ncol=}\DecValTok{2}\NormalTok{, }\DataTypeTok{byrow=}\OtherTok{TRUE}\NormalTok{) }\CommentTok{# on défini X}
\NormalTok{X<-}\KeywordTok{as.data.frame}\NormalTok{(A) }\CommentTok{# changement de format}
\KeywordTok{colnames}\NormalTok{(X)<-}\KeywordTok{c}\NormalTok{(}\StringTok{"var1"}\NormalTok{, }\StringTok{"var2"}\NormalTok{) }\CommentTok{# on nomme les colonnes}
\KeywordTok{rownames}\NormalTok{(X)<-}\KeywordTok{c}\NormalTok{(}\StringTok{"g123"}\NormalTok{,}\StringTok{"w4"}\NormalTok{,}\StringTok{"g567"}\NormalTok{) }\CommentTok{# on nomme les lignes}
\NormalTok{X }\CommentTok{# on affiche X}
\end{Highlighting}
\end{Shaded}

\begin{verbatim}
##          var1      var2
## g123 1.000000 0.6666667
## w4   3.000000 2.0000000
## g567 3.666667 3.6666667
\end{verbatim}

\begin{Shaded}
\begin{Highlighting}[]
\CommentTok{# Distance euclidienne}
\NormalTok{dist <-}\StringTok{ }\KeywordTok{dist}\NormalTok{(X, }\DataTypeTok{method =} \StringTok{"euclidean"}\NormalTok{)}
\NormalTok{dist}
\end{Highlighting}
\end{Shaded}

\begin{verbatim}
##          g123       w4
## w4   2.403701         
## g567 4.013865 1.795055
\end{verbatim}

\begin{Shaded}
\begin{Highlighting}[]
\CommentTok{##La matrice des distances au carré}
\NormalTok{dist}\OperatorTok{^}\DecValTok{2}
\end{Highlighting}
\end{Shaded}

\begin{verbatim}
##           g123        w4
## w4    5.777778          
## g567 16.111111  3.222222
\end{verbatim}

\begin{Shaded}
\begin{Highlighting}[]
\CommentTok{#La matrice des sauts de ward avec la distance au carré est:}
\NormalTok{ward1<-(dist}\OperatorTok{^}\DecValTok{2}\NormalTok{)}\OperatorTok{/}\DecValTok{8}
\NormalTok{ward1}
\end{Highlighting}
\end{Shaded}

\begin{verbatim}
##           g123        w4
## w4   0.7222222          
## g567 2.0138889 0.4027778
\end{verbatim}

\begin{Shaded}
\begin{Highlighting}[]
\CommentTok{##La plus petite perte d'inertie inter est entre g567 et w4}
\NormalTok{pertemin1=}\KeywordTok{min}\NormalTok{(ward1)}
\KeywordTok{fviz_dist}\NormalTok{(dist, }\DataTypeTok{lab_size =} \DecValTok{8}\NormalTok{)}
\end{Highlighting}
\end{Shaded}

\includegraphics{TP3_Dellouve_Gazzo_files/figure-latex/unnamed-chunk-2-5.pdf}

on a donc un groupe : w1 w2 w3

et un autre : w4 w5 w6 w7

\begin{Shaded}
\begin{Highlighting}[]
\NormalTok{A=}\KeywordTok{matrix}\NormalTok{(}\KeywordTok{c}\NormalTok{(}\DecValTok{1}\NormalTok{,}\DecValTok{0}\NormalTok{,}\DecValTok{2}\NormalTok{,}\DecValTok{1}\NormalTok{,}\DecValTok{0}\NormalTok{,}\DecValTok{1}\NormalTok{,}\DecValTok{3}\NormalTok{,}\DecValTok{2}\NormalTok{,}\DecValTok{4}\NormalTok{,}\DecValTok{3}\NormalTok{,}\DecValTok{3}\NormalTok{,}\DecValTok{4}\NormalTok{,}\DecValTok{4}\NormalTok{,}\DecValTok{4}\NormalTok{),}\DataTypeTok{ncol=}\DecValTok{2}\NormalTok{, }\DataTypeTok{byrow=}\OtherTok{TRUE}\NormalTok{) }\CommentTok{# on défini X}
\NormalTok{X=}\KeywordTok{as.data.frame}\NormalTok{(A) }\CommentTok{# changement de format}
\KeywordTok{colnames}\NormalTok{(X)=}\KeywordTok{c}\NormalTok{(}\StringTok{"var1"}\NormalTok{, }\StringTok{"var2"}\NormalTok{) }\CommentTok{# on nomme les colonnes}
\KeywordTok{rownames}\NormalTok{(X)=}\KeywordTok{c}\NormalTok{(}\StringTok{"w1"}\NormalTok{, }\StringTok{"w2"}\NormalTok{, }\StringTok{"w3"}\NormalTok{,}\StringTok{"w4"}\NormalTok{,}\StringTok{"w5"}\NormalTok{,}\StringTok{"w6"}\NormalTok{,}\StringTok{"w7"}\NormalTok{) }\CommentTok{# on nomme les lignes}

\CommentTok{# Distance euclidienne}
\CommentTok{## CAH sur des données brutes avec hclust ou eclust}
\CommentTok{# Distance euclidienne}
\NormalTok{dist <-}\StringTok{ }\KeywordTok{dist}\NormalTok{(X, }\DataTypeTok{method =} \StringTok{"euclidean"}\NormalTok{)}
\CommentTok{# CAH avec Ward (option ward.D2 avec les distances au carré ou ward avec la distance)}
\NormalTok{hc <-}\StringTok{ }\KeywordTok{hclust}\NormalTok{(dist, }\DataTypeTok{method =} \StringTok{"ward.D2"}\NormalTok{)}
\CommentTok{# Les regroupements pas à pas}
\NormalTok{hc}\OperatorTok{$}\NormalTok{merge}
\end{Highlighting}
\end{Shaded}

\begin{verbatim}
##      [,1] [,2]
## [1,]   -5   -7
## [2,]   -6    1
## [3,]   -1   -2
## [4,]   -3    3
## [5,]   -4    2
## [6,]    4    5
\end{verbatim}

\begin{Shaded}
\begin{Highlighting}[]
\CommentTok{# La hauteur des branches est la distance}
\NormalTok{hc}\OperatorTok{$}\NormalTok{height}
\end{Highlighting}
\end{Shaded}

\begin{verbatim}
## [1] 1.000000 1.290994 1.414214 1.825742 2.198484 6.656540
\end{verbatim}

\begin{Shaded}
\begin{Highlighting}[]
\CommentTok{# Arbre}
\KeywordTok{plot}\NormalTok{(hc, }\DataTypeTok{cex =} \FloatTok{0.5}\NormalTok{)}
\end{Highlighting}
\end{Shaded}

\includegraphics{TP3_Dellouve_Gazzo_files/figure-latex/unnamed-chunk-3-1.pdf}

\begin{Shaded}
\begin{Highlighting}[]
\NormalTok{cut<-}\KeywordTok{cutree}\NormalTok{(hc,}\DataTypeTok{k=}\DecValTok{2}\NormalTok{)}
\KeywordTok{print}\NormalTok{(}\KeywordTok{sort}\NormalTok{(cut))}
\end{Highlighting}
\end{Shaded}

\begin{verbatim}
## w1 w2 w3 w4 w5 w6 w7 
##  1  1  1  2  2  2  2
\end{verbatim}

On voit que si on veut deux classes, on obtient un premier groupe
composé de \(\omega_1\),\(\omega_2\) et \(\omega_3\); ainsi qu'un
deuxième groupe composé de \(\omega_4\),\(\omega_5\),\(\omega_6\) et
\(\omega_7\).

\begin{Shaded}
\begin{Highlighting}[]
\CommentTok{##CAH avec eclust de factoextra}
\CommentTok{#hc2 <- eclust(X, k.max=3,"hclust")}
\CommentTok{#hc2$merge}
\CommentTok{#hc2$height}
\CommentTok{# dendrogamme}
\CommentTok{#fviz_dend(hc2) }
\CommentTok{## Couper le dendogramme pour avoir 2 classes}
\CommentTok{#coupe <- hcut(X, k = 2, hc_method = "complete")}
\CommentTok{# Visualiser le dendrogramme avec les 2 classes}
\CommentTok{#fviz_dend(coupe, show_labels = FALSE, rect = TRUE)}
\CommentTok{# Visualiser les classes en fonction des variables}
\CommentTok{#fviz_cluster(coupe, ellipse.type = "convex")}
\CommentTok{# Visualize silhouhette information}
\CommentTok{#fviz_silhouette(coupe)}
\end{Highlighting}
\end{Shaded}

\#Question 2

Faisons la classification avec K = 2 classes et des graines initiales
\(\omega_5\) et \(\omega_7\).

\begin{Shaded}
\begin{Highlighting}[]
\CommentTok{# Distance euclidienne}

\NormalTok{A<-}\KeywordTok{matrix}\NormalTok{(}\KeywordTok{c}\NormalTok{(}\DecValTok{1}\NormalTok{,}\DecValTok{0}\NormalTok{,}\DecValTok{2}\NormalTok{,}\DecValTok{1}\NormalTok{,}\DecValTok{0}\NormalTok{,}\DecValTok{1}\NormalTok{,}\DecValTok{3}\NormalTok{,}\DecValTok{2}\NormalTok{,}\DecValTok{4}\NormalTok{,}\DecValTok{3}\NormalTok{,}\DecValTok{3}\NormalTok{,}\DecValTok{4}\NormalTok{,}\DecValTok{4}\NormalTok{,}\DecValTok{4}\NormalTok{),}\DataTypeTok{ncol=}\DecValTok{2}\NormalTok{, }\DataTypeTok{byrow=}\OtherTok{TRUE}\NormalTok{) }\CommentTok{# on défini X}
\NormalTok{X<-}\KeywordTok{as.data.frame}\NormalTok{(A) }\CommentTok{# changement de format}
\KeywordTok{colnames}\NormalTok{(X)<-}\KeywordTok{c}\NormalTok{(}\StringTok{"var1"}\NormalTok{, }\StringTok{"var2"}\NormalTok{) }\CommentTok{# on nomme les colonnes}
\KeywordTok{rownames}\NormalTok{(X)<-}\KeywordTok{c}\NormalTok{(}\StringTok{"w1"}\NormalTok{, }\StringTok{"w2"}\NormalTok{, }\StringTok{"w3"}\NormalTok{,}\StringTok{"w4"}\NormalTok{,}\StringTok{"w5"}\NormalTok{,}\StringTok{"w6"}\NormalTok{,}\StringTok{"w7"}\NormalTok{) }\CommentTok{# on nomme les lignes}
\NormalTok{X }\CommentTok{# on affiche X}
\end{Highlighting}
\end{Shaded}

\begin{verbatim}
##    var1 var2
## w1    1    0
## w2    2    1
## w3    0    1
## w4    3    2
## w5    4    3
## w6    3    4
## w7    4    4
\end{verbatim}

\begin{Shaded}
\begin{Highlighting}[]
\NormalTok{dist <-}\StringTok{ }\KeywordTok{dist}\NormalTok{(X, }\DataTypeTok{method =} \StringTok{"euclidean"}\NormalTok{)}
\NormalTok{dist}
\end{Highlighting}
\end{Shaded}

\begin{verbatim}
##          w1       w2       w3       w4       w5       w6
## w2 1.414214                                             
## w3 1.414214 2.000000                                    
## w4 2.828427 1.414214 3.162278                           
## w5 4.242641 2.828427 4.472136 1.414214                  
## w6 4.472136 3.162278 4.242641 2.000000 1.414214         
## w7 5.000000 3.605551 5.000000 2.236068 1.000000 1.000000
\end{verbatim}

\begin{Shaded}
\begin{Highlighting}[]
\CommentTok{#g1  : w1 w2 w3 w4 w5}
\CommentTok{#g2  : w6 w7 }

\NormalTok{A<-}\KeywordTok{matrix}\NormalTok{(}\KeywordTok{c}\NormalTok{(}\DecValTok{1}\NormalTok{,}\DecValTok{0}\NormalTok{,}\DecValTok{2}\NormalTok{,}\DecValTok{1}\NormalTok{,}\DecValTok{0}\NormalTok{,}\DecValTok{1}\NormalTok{,}\DecValTok{3}\NormalTok{,}\DecValTok{2}\NormalTok{,}\DecValTok{4}\NormalTok{,}\DecValTok{3}\NormalTok{,}\DecValTok{3}\NormalTok{,}\DecValTok{4}\NormalTok{,}\DecValTok{4}\NormalTok{,}\DecValTok{4}\NormalTok{,}\DecValTok{2}\NormalTok{,}\FloatTok{7.}\OperatorTok{/}\DecValTok{5}\NormalTok{,}\FloatTok{3.5}\NormalTok{,}\DecValTok{4}\NormalTok{),}\DataTypeTok{ncol=}\DecValTok{2}\NormalTok{, }\DataTypeTok{byrow=}\OtherTok{TRUE}\NormalTok{) }\CommentTok{# on défini X}
\NormalTok{X<-}\KeywordTok{as.data.frame}\NormalTok{(A) }\CommentTok{# changement de format}
\KeywordTok{colnames}\NormalTok{(X)<-}\KeywordTok{c}\NormalTok{(}\StringTok{"var1"}\NormalTok{, }\StringTok{"var2"}\NormalTok{) }\CommentTok{# on nomme les colonnes}
\KeywordTok{rownames}\NormalTok{(X)<-}\KeywordTok{c}\NormalTok{(}\StringTok{"w1"}\NormalTok{, }\StringTok{"w2"}\NormalTok{, }\StringTok{"w3"}\NormalTok{,}\StringTok{"w4"}\NormalTok{,}\StringTok{"w5"}\NormalTok{,}\StringTok{"w6"}\NormalTok{,}\StringTok{"w7"}\NormalTok{,}\StringTok{"g1"}\NormalTok{,}\StringTok{"g2"}\NormalTok{) }\CommentTok{# on nomme les lignes}
\NormalTok{X }\CommentTok{# on affiche X}
\end{Highlighting}
\end{Shaded}

\begin{verbatim}
##    var1 var2
## w1  1.0  0.0
## w2  2.0  1.0
## w3  0.0  1.0
## w4  3.0  2.0
## w5  4.0  3.0
## w6  3.0  4.0
## w7  4.0  4.0
## g1  2.0  1.4
## g2  3.5  4.0
\end{verbatim}

\begin{Shaded}
\begin{Highlighting}[]
\CommentTok{# Distance euclidienne}
\NormalTok{dist <-}\StringTok{ }\KeywordTok{dist}\NormalTok{(X, }\DataTypeTok{method =} \StringTok{"euclidean"}\NormalTok{)}
\NormalTok{dist}
\end{Highlighting}
\end{Shaded}

\begin{verbatim}
##          w1       w2       w3       w4       w5       w6       w7       g1
## w2 1.414214                                                               
## w3 1.414214 2.000000                                                      
## w4 2.828427 1.414214 3.162278                                             
## w5 4.242641 2.828427 4.472136 1.414214                                    
## w6 4.472136 3.162278 4.242641 2.000000 1.414214                           
## w7 5.000000 3.605551 5.000000 2.236068 1.000000 1.000000                  
## g1 1.720465 0.400000 2.039608 1.166190 2.561250 2.785678 3.280244         
## g2 4.716991 3.354102 4.609772 2.061553 1.118034 0.500000 0.500000 3.001666
\end{verbatim}

\begin{Shaded}
\begin{Highlighting}[]
\CommentTok{# w5 passe dans le groupe 2}

\NormalTok{A<-}\KeywordTok{matrix}\NormalTok{(}\KeywordTok{c}\NormalTok{(}\DecValTok{1}\NormalTok{,}\DecValTok{0}\NormalTok{,}\DecValTok{2}\NormalTok{,}\DecValTok{1}\NormalTok{,}\DecValTok{0}\NormalTok{,}\DecValTok{1}\NormalTok{,}\DecValTok{3}\NormalTok{,}\DecValTok{2}\NormalTok{,}\DecValTok{4}\NormalTok{,}\DecValTok{3}\NormalTok{,}\DecValTok{3}\NormalTok{,}\DecValTok{4}\NormalTok{,}\DecValTok{4}\NormalTok{,}\DecValTok{4}\NormalTok{,}\FloatTok{6.}\OperatorTok{/}\DecValTok{4}\NormalTok{,}\DecValTok{1}\NormalTok{,}\DecValTok{11}\OperatorTok{/}\DecValTok{3}\NormalTok{,}\DecValTok{11}\OperatorTok{/}\DecValTok{3}\NormalTok{),}\DataTypeTok{ncol=}\DecValTok{2}\NormalTok{, }\DataTypeTok{byrow=}\OtherTok{TRUE}\NormalTok{) }\CommentTok{# on défini X}
\NormalTok{X<-}\KeywordTok{as.data.frame}\NormalTok{(A) }\CommentTok{# changement de format}
\KeywordTok{colnames}\NormalTok{(X)<-}\KeywordTok{c}\NormalTok{(}\StringTok{"var1"}\NormalTok{, }\StringTok{"var2"}\NormalTok{) }\CommentTok{# on nomme les colonnes}
\KeywordTok{rownames}\NormalTok{(X)<-}\KeywordTok{c}\NormalTok{(}\StringTok{"w1"}\NormalTok{, }\StringTok{"w2"}\NormalTok{, }\StringTok{"w3"}\NormalTok{,}\StringTok{"w4"}\NormalTok{,}\StringTok{"w5"}\NormalTok{,}\StringTok{"w6"}\NormalTok{,}\StringTok{"w7"}\NormalTok{,}\StringTok{"g1"}\NormalTok{,}\StringTok{"g2"}\NormalTok{) }\CommentTok{# on nomme les lignes}
\NormalTok{X }\CommentTok{# on affiche X}
\end{Highlighting}
\end{Shaded}

\begin{verbatim}
##        var1     var2
## w1 1.000000 0.000000
## w2 2.000000 1.000000
## w3 0.000000 1.000000
## w4 3.000000 2.000000
## w5 4.000000 3.000000
## w6 3.000000 4.000000
## w7 4.000000 4.000000
## g1 1.500000 1.000000
## g2 3.666667 3.666667
\end{verbatim}

\begin{Shaded}
\begin{Highlighting}[]
\CommentTok{# Distance euclidienne}
\NormalTok{dist <-}\StringTok{ }\KeywordTok{dist}\NormalTok{(X, }\DataTypeTok{method =} \StringTok{"euclidean"}\NormalTok{)}
\NormalTok{dist}
\end{Highlighting}
\end{Shaded}

\begin{verbatim}
##           w1        w2        w3        w4        w5        w6        w7
## w2 1.4142136                                                            
## w3 1.4142136 2.0000000                                                  
## w4 2.8284271 1.4142136 3.1622777                                        
## w5 4.2426407 2.8284271 4.4721360 1.4142136                              
## w6 4.4721360 3.1622777 4.2426407 2.0000000 1.4142136                    
## w7 5.0000000 3.6055513 5.0000000 2.2360680 1.0000000 1.0000000          
## g1 1.1180340 0.5000000 1.5000000 1.8027756 3.2015621 3.3541020 3.9051248
## g2 4.5338235 3.1446604 4.5338235 1.7950549 0.7453560 0.7453560 0.4714045
##           g1
## w2          
## w3          
## w4          
## w5          
## w6          
## w7          
## g1          
## g2 3.4359214
\end{verbatim}

\begin{Shaded}
\begin{Highlighting}[]
\CommentTok{#w4 passe dans le groupe 2}

\NormalTok{A<-}\KeywordTok{matrix}\NormalTok{(}\KeywordTok{c}\NormalTok{(}\DecValTok{1}\NormalTok{,}\DecValTok{0}\NormalTok{,}\DecValTok{2}\NormalTok{,}\DecValTok{1}\NormalTok{,}\DecValTok{0}\NormalTok{,}\DecValTok{1}\NormalTok{,}\DecValTok{3}\NormalTok{,}\DecValTok{2}\NormalTok{,}\DecValTok{4}\NormalTok{,}\DecValTok{3}\NormalTok{,}\DecValTok{3}\NormalTok{,}\DecValTok{4}\NormalTok{,}\DecValTok{4}\NormalTok{,}\DecValTok{4}\NormalTok{,}\DecValTok{1}\NormalTok{,}\FloatTok{2.}\OperatorTok{/}\DecValTok{3}\NormalTok{,}\FloatTok{14.}\OperatorTok{/}\DecValTok{4}\NormalTok{,}\FloatTok{13.}\OperatorTok{/}\DecValTok{4}\NormalTok{),}\DataTypeTok{ncol=}\DecValTok{2}\NormalTok{, }\DataTypeTok{byrow=}\OtherTok{TRUE}\NormalTok{) }\CommentTok{# on défini X}
\NormalTok{X<-}\KeywordTok{as.data.frame}\NormalTok{(A) }\CommentTok{# changement de format}
\KeywordTok{colnames}\NormalTok{(X)<-}\KeywordTok{c}\NormalTok{(}\StringTok{"var1"}\NormalTok{, }\StringTok{"var2"}\NormalTok{) }\CommentTok{# on nomme les colonnes}
\KeywordTok{rownames}\NormalTok{(X)<-}\KeywordTok{c}\NormalTok{(}\StringTok{"w1"}\NormalTok{, }\StringTok{"w2"}\NormalTok{, }\StringTok{"w3"}\NormalTok{,}\StringTok{"w4"}\NormalTok{,}\StringTok{"w5"}\NormalTok{,}\StringTok{"w6"}\NormalTok{,}\StringTok{"w7"}\NormalTok{,}\StringTok{"g1"}\NormalTok{,}\StringTok{"g2"}\NormalTok{) }\CommentTok{# on nomme les lignes}
\NormalTok{X }\CommentTok{# on affiche X}
\end{Highlighting}
\end{Shaded}

\begin{verbatim}
##    var1      var2
## w1  1.0 0.0000000
## w2  2.0 1.0000000
## w3  0.0 1.0000000
## w4  3.0 2.0000000
## w5  4.0 3.0000000
## w6  3.0 4.0000000
## w7  4.0 4.0000000
## g1  1.0 0.6666667
## g2  3.5 3.2500000
\end{verbatim}

\begin{Shaded}
\begin{Highlighting}[]
\CommentTok{# Distance euclidienne}
\NormalTok{dist <-}\StringTok{ }\KeywordTok{dist}\NormalTok{(X, }\DataTypeTok{method =} \StringTok{"euclidean"}\NormalTok{)}
\NormalTok{dist}
\end{Highlighting}
\end{Shaded}

\begin{verbatim}
##           w1        w2        w3        w4        w5        w6        w7
## w2 1.4142136                                                            
## w3 1.4142136 2.0000000                                                  
## w4 2.8284271 1.4142136 3.1622777                                        
## w5 4.2426407 2.8284271 4.4721360 1.4142136                              
## w6 4.4721360 3.1622777 4.2426407 2.0000000 1.4142136                    
## w7 5.0000000 3.6055513 5.0000000 2.2360680 1.0000000 1.0000000          
## g1 0.6666667 1.0540926 1.0540926 2.4037009 3.8005848 3.8873013 4.4845413
## g2 4.1003049 2.7041635 4.1608292 1.3462912 0.5590170 0.9013878 0.9013878
##           g1
## w2          
## w3          
## w4          
## w5          
## w6          
## w7          
## g1          
## g2 3.5949424
\end{verbatim}

On retrouve le même résultat.

\begin{Shaded}
\begin{Highlighting}[]
\NormalTok{A<-}\KeywordTok{matrix}\NormalTok{(}\KeywordTok{c}\NormalTok{(}\DecValTok{1}\NormalTok{,}\DecValTok{0}\NormalTok{,}\DecValTok{2}\NormalTok{,}\DecValTok{1}\NormalTok{,}\DecValTok{0}\NormalTok{,}\DecValTok{1}\NormalTok{,}\DecValTok{3}\NormalTok{,}\DecValTok{2}\NormalTok{,}\DecValTok{4}\NormalTok{,}\DecValTok{3}\NormalTok{,}\DecValTok{3}\NormalTok{,}\DecValTok{4}\NormalTok{,}\DecValTok{4}\NormalTok{,}\DecValTok{4}\NormalTok{),}\DataTypeTok{ncol=}\DecValTok{2}\NormalTok{, }\DataTypeTok{byrow=}\OtherTok{TRUE}\NormalTok{) }\CommentTok{# on défini X}
\NormalTok{X<-}\KeywordTok{as.data.frame}\NormalTok{(A) }\CommentTok{# changement de format}
\KeywordTok{colnames}\NormalTok{(X)<-}\KeywordTok{c}\NormalTok{(}\StringTok{"var1"}\NormalTok{, }\StringTok{"var2"}\NormalTok{) }\CommentTok{# on nomme les colonnes}
\KeywordTok{rownames}\NormalTok{(X)<-}\KeywordTok{c}\NormalTok{(}\StringTok{"w1"}\NormalTok{, }\StringTok{"w2"}\NormalTok{, }\StringTok{"w3"}\NormalTok{,}\StringTok{"w4"}\NormalTok{,}\StringTok{"w5"}\NormalTok{,}\StringTok{"w6"}\NormalTok{,}\StringTok{"w7"}\NormalTok{) }\CommentTok{# on nomme les lignes}
\NormalTok{X }\CommentTok{# on affiche X}
\end{Highlighting}
\end{Shaded}

\begin{verbatim}
##    var1 var2
## w1    1    0
## w2    2    1
## w3    0    1
## w4    3    2
## w5    4    3
## w6    3    4
## w7    4    4
\end{verbatim}

\begin{Shaded}
\begin{Highlighting}[]
\CommentTok{##K-means}
\NormalTok{km <-}\StringTok{ }\KeywordTok{eclust}\NormalTok{(X, }\StringTok{"kmeans"}\NormalTok{, }\DataTypeTok{k=}\DecValTok{2}\NormalTok{)}
\end{Highlighting}
\end{Shaded}

\includegraphics{TP3_Dellouve_Gazzo_files/figure-latex/unnamed-chunk-6-1.pdf}

\begin{Shaded}
\begin{Highlighting}[]
\NormalTok{km}\OperatorTok{$}\NormalTok{cluster}
\end{Highlighting}
\end{Shaded}

\begin{verbatim}
## w1 w2 w3 w4 w5 w6 w7 
##  2  2  2  1  1  1  1
\end{verbatim}

\begin{Shaded}
\begin{Highlighting}[]
\CommentTok{# Visualisation}
\KeywordTok{fviz_cluster}\NormalTok{(km, }\DataTypeTok{geom =} \StringTok{"text"}\NormalTok{)}
\end{Highlighting}
\end{Shaded}

\includegraphics{TP3_Dellouve_Gazzo_files/figure-latex/unnamed-chunk-6-2.pdf}

\begin{Shaded}
\begin{Highlighting}[]
\CommentTok{## Choix de k}
\NormalTok{km <-}\StringTok{ }\KeywordTok{eclust}\NormalTok{(X, }\StringTok{"kmeans"}\NormalTok{, }\DataTypeTok{nstart =} \DecValTok{2}\NormalTok{, }\DataTypeTok{k.max=}\DecValTok{3}\NormalTok{)}
\end{Highlighting}
\end{Shaded}

\includegraphics{TP3_Dellouve_Gazzo_files/figure-latex/unnamed-chunk-6-3.pdf}

\begin{Shaded}
\begin{Highlighting}[]
\CommentTok{##Gap}
\KeywordTok{fviz_gap_stat}\NormalTok{(km}\OperatorTok{$}\NormalTok{gap_stat)}
\end{Highlighting}
\end{Shaded}

\includegraphics{TP3_Dellouve_Gazzo_files/figure-latex/unnamed-chunk-6-4.pdf}

\end{document}
