% !TEX encoding = IsoLatin9
\documentclass[a4,12pt,french]{article}
\usepackage[frenchb]{babel}
\usepackage[utf8]{inputenc}
\usepackage{babel,fontenc,indentfirst,a4,longtable,ulem}
\usepackage[dvips]{graphicx}
\marginparwidth 0pt
\oddsidemargin -1cm
\evensidemargin 0pt
\marginparsep 0pt
\topmargin -1cm
\textwidth 17.5cm
\textheight 24cm
\parindent 0mm

\newcommand{\biz}{\begin{itemize}}
\newcommand{\eiz}{\end{itemize}}
\newcommand{\ben}{\begin{enumerate}}
\newcommand{\een}{\end{enumerate}}
\newcommand{\btab}{\begin{tabular}}
\newcommand{\etab}{\end{tabular}}
\newcommand{\barr}{\begin{array}}
\newcommand{\earr}{\end{array}}
\newcommand{\bcent}{\begin{center}}
\newcommand{\ecent}{\end{center}}
\newcommand{\ovl}{\overline}
\newcommand{\bqa}{\begin{eqnarray}}
\newcommand{\eqa}{\end{eqnarray}}
\newcommand{\bqas}{\begin{eqnarray*}}
\newcommand{\eqas}{\end{eqnarray*}}
\newcommand{\dps}{\displaystyle}
\newcommand{\psep}{\quad ; \quad}
\newcommand{\intd}{\int \! \! \int}
\newcommand{\noi}{\noindent}
\newcommand{\exo}[1]{\vskip 0.2cm\noi\underline{EXERCICE #1}\vskip 0.1cm\noi}

\usepackage{Sweave}
\begin{document}
\input{CorrectionTP1-concordance}
\noi \begin{tabular}[t]{lr}
\textbf{Statistique Spatiale} &\hskip 6cm \textbf{Ann\'{e}e Universitaire 2020/2021} \\
%\textbf{Statistique Spatiale} &  \\
\end{tabular}

\noi
\hrulefill

\begin{center}
\textbf{VARIOGRAMMES, KRIGEAGE}
\end{center}
\vskip 0.5cm
Le logiciel utilis\'e est R, avec les packages {\it geoR} et {\it fields}.\\

\begin{enumerate}
\item
{\bf Donn\'ees simul\'ees }
On dispose d'une simulation d'un champ gaussien sur un carré 101$\times$101.
On extrait 100 points du carré et on va reconstruire l'image sur le carré par krigeage.
Pour cela il faudra d'abord ajuster un variogramme sur une fonction choisie.
On comparera les résultats obtenus avec différents variogrammes à l'image originale.

\begin{enumerate}
\item {Visualisation}

Le fichier {\it simu1.dat} contient une simulation d'un processus gaussien sur une grille
101$\times$101.\\
Charger le fichier  avec la fonction  \textit{read.table}.\\
Les 3 variables: $x$ abcisse, $y$ ordonnée et $z$ valeur du champ sont en colonnes.\\
Les fonctions \textit{summary} et \textit{hist} donnent les statistiques élémentaires et l'histogramme d'une variable.\\
Visualiser le champ
  avec la fonction \textit{image.plot}. Il faudra tout d'abord transformer le vecteur $z$ en
une matrice avec la fonction \textit{matrix}.


